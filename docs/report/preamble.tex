\usepackage[tmargin=2cm,rmargin=1in,lmargin=1in,margin=1in,bmargin=2cm,footskip=.2in]{geometry}
\usepackage[font=small,labelfont=bf]{caption}
\usepackage{graphicx}
\usepackage{listings}
\usepackage{multirow}
\usepackage{color}
\usepackage[T1]{fontenc}
\usepackage{setspace}
\usepackage[ngerman]{babel}
\usepackage{tabularx}
\usepackage[nohyphen]{underscore}
\usepackage{longtable} % Für Tabellen, die über mehrere Seiten gehen können
\usepackage{ragged2e} % Für \RaggedRight
\usepackage{fancyhdr}
\usepackage{xcolor}
\usepackage{hyperref}
\usepackage{array} % Für erweiterte Tabellenspalten-Optionen
\usepackage{xcolor} % Enables color functionality
\usepackage{colortbl} % Provides \rowcolor command
\usepackage{ltablex}

\definecolor{codecomment}{rgb}{0,0.6,0}
\definecolor{codelinenumber}{rgb}{0.5,0.5,0.5}
\definecolor{codestring}{rgb}{0.58,0,0.82}
\definecolor{codebg}{rgb}{0.95, 0.95, 0.95}

\definecolor{storyblue}{RGB}{0, 102, 204}
\definecolor{storygreen}{RGB}{0, 153, 51}
\definecolor{tableheader}{RGB}{230, 230, 230} % Heller Grauton für Tabellenkopf
\definecolor{weather-flosscast}{RGB}{200, 220, 240} % Example: a light blue color

\makeatletter
\def\@maketitle{
  \null
  \vskip 4cm
  \begin{center}%
    {%
      \doublespacing
      \Large\bfseries \@title \par
    }
    \vskip 4em
    {\normalsize
      \lineskip .5em
      \begin{tabular}[t]{c}
        \@author
    \end{tabular}\par}
    \vskip 4em
    {\normalsize \textbf{\@date}}
    \vskip 1cm
    \url{https://github.com/bzundel/weather-flosscast}
  \end{center}%
  \par
  \vskip 1.5em}
\makeatother

% Attribution to David Carlisle (@david-carlisle) on TeX StackExchange
% https://tex.stackexchange.com/questions/618849/how-to-create-subsubsubsection
\makeatletter
\renewcommand\paragraph{\@startsection{paragraph}{4}{\z@}%
                                     {-3.25ex\@plus -1ex \@minus -.2ex}%
                                     {1.5ex \@plus .2ex}%
                                     {\normalfont\normalsize\bfseries}}
\setcounter{secnumdepth}{4}
\makeatother

\lstset{
  backgroundcolor=\color{codebg},
  basicstyle=\ttfamily\footnotesize,
  breakatwhitespace=true,
  breaklines=true,
  captionpos=b,
  commentstyle=\color{codecomment},
  escapeinside={\%*}{*)},
  extendedchars=true,
  frame=lines,
  keepspaces=true,
  keywordstyle=\color{blue},
  numbers=left,
  numbersep=10pt,
  numberstyle=\tiny\color{codelinenumber},
  rulecolor=\color{black},
  showspaces=false,
  showstringspaces=false,
  showtabs=false,
  stepnumber=1,
  stringstyle=\color{codestring},
  tabsize=4,
  captionpos=top
}
